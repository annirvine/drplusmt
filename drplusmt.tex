%
% File naaclhlt2013.tex
%

\documentclass[11pt,letterpaper]{article}
\usepackage{naaclhlt2013}
\usepackage{times}
\usepackage{latexsym}
\setlength\titlebox{6.5cm}    % Expanding the titlebox

\title{Distributed Representations for Lexicon Induction}

%\author{Author 1\\
%	  \And
%	Author 2\\}

\date{}

\begin{document}
\maketitle
\begin{abstract}
write me
\end{abstract}

\section{Introduction}
write me

\section{Previous Work}
write me

\subsection{Model}
write me

\subsection{Experiments}

Table \ref{accresults} shows performance on the lexicon induction task. 
The old contextual score uses a dictionary based on the intersection alignments over the training data for each language to project context vectors.
The distributed representations use an interaction matrix defined also by the intersection alignments over the training data for each language.
Both models use the same tokenization of all of the monolingual data that we have available for each language, which is taken from web crawls and Wikipedia.
Evaluation is over {\it all} word types in the development set  for each language.


\begin{table}
\begin{center}
\begin{tabular}{|l|c|c|c|}
\hline
& Top-1 & Top-10 & Top-100 \\
\hline
\multicolumn{4}{|l|}{Tamil}  \\
\hline
Old-Contextual & 2.32 & 8.38 & 25.44 \\ 
Distrib Rep L2 Dist & 15.50 & 17.77 & 20.44 \\
\hline
\multicolumn{4}{|l|}{Bengali}  \\
\hline
Old-Contextual & 3.91 & 12.39 & 30.53 \\
Distrib Rep L2 Dist & 24.01 & 25.86 & 28.01 \\
\hline
\multicolumn{4}{|l|}{Hindi}  \\
\hline
Old-Contextual & 5.22 & 14.72 & 34.31 \\
Distrib Rep L2 Dist & 33.93 & 37.64 & 42.00 \\
\hline
\end{tabular}
\end{center}
\caption{Comparison of performance of old definition of contextual similarity with new distributed representations model}\label{accresults}
\end{table}

% Note to Anni: eval script commands used to generate the above
% old: 
% python evalout.py ../../originalCosComparison/ta/intersection/output/context.scored ../getAlignmentBasedDictionaries/growdiagfinaltranslations.plusbigall.ta ~/inducePhraseTable/LIMT/Experiments/IndianLangsCorpus/ta-en/dev.allwords cr tempoutput ta
% new: 
% python evalout.py ../dodoOutputs/111912outs/devdevtest.ta.epoch49 ../getAlignmentBasedDictionaries/growdiagfinaltranslations.plusbigall.ta ~/inducePhraseTable/LIMT/Experiments/IndianLangsCorpus/ta-en/dev.allwords dr tempoutput ta





\bibliographystyle{naaclhlt2013}
\bibliography{bibfile}


\end{document}
